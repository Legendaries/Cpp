%http://detexify.kirelabs.org/classify.html
%http://www.cs.uml.edu/~jannunzi/91.404/classNotes/ch3/3.1%20Summation%20Formulas%20and%20Properties.html
%http://www.cs.uml.edu/~jannunzi/91.404/classNotes/ch3/3.1%20Summation%20Formulas%20and%20Properties.html
\documentclass{article}
%\documentclass[titlepage]{article}

\usepackage{graphicx}
\usepackage{geometry}
\geometry{letterpaper}
\usepackage{amssymb,amsmath}
\usepackage{epstopdf}					
\usepackage{color}	
\usepackage{ragged2e}
\usepackage{setspace}
\usepackage{etoolbox}
\usepackage{lipsum}

\newcommand{\blu}[1]{\textcolor{blue}{#1}}
\newcommand{\red}[1]{\textcolor{red}{#1}}
\definecolor{light-gray}{gray}{0.8}
\definecolor{orange}{rgb}{1,0.5,0}
\newcommand{\gray}[1]{\textcolor{light-gray}{#1}}

\title{FVCC Linear Algebra\\ Proof of The Distributive Property of Matrix Multiplication Over Addition  \\ M-221}
\author{Daniel Church}				
\date{\today}
\begin{document}
\maketitle
\noindent
\rule[4mm]{17cm}{2pt}
\begin{tabular}{l r}
Date Due: & \today \\
Instructor: & M. Severino
\end{tabular}\\
\rule[4mm]{17cm}{2pt}

\newpage
{\centering
\LARGE Proof of The Distributive Property of Matrix Multiplication Over Addition \par
}
\setstretch{2.5}
\begin{tabular}{@{}l l l @{\hskip 2.5in}l}
To Show: $A(B+C)$ & = & $AB + AC$ & 
\end{tabular}

Given $A, B$ and $C$ are matrices of size $m$ x $n$, $n$ x $p$ and $n$ x $p$ respectively,

\begin{tabular}{@{}l l l @{\hskip 1.5in}l}
Let $D$ & = & $A+B$ & By Addition of Matrices\\
Then, $A(B+C)$ & = & $AD$ & By Substituion\\
$(AD)_{ij}$ & = & $\displaystyle\sum_{k=1}^{n} a_{ik}d_{kj}$ & By Def. of Matrix Multiplication\\
 & = & $\displaystyle\sum_{k=1}^{n} a_{ik}(b_{kj}+c_{kj})$ & By Substituion\\
 & = & $\displaystyle\sum_{k=1}^{n} (a_{ik}b_{kj}+a_{ik}c_{kj})$ & By The Distrubutive Property of Scalars\\
& = & $\displaystyle\sum_{k=1}^{n} a_{ik}b_{kj} + \displaystyle\sum_{k=1}^{n} a_{ik}c_{kj}$ & By Summation Linearity\\
& = & $(AB)_{ij}+(AC)_{ij}$ & By Def. of Matrix Multiplication\\
& = & $(AB+AC)_{ij}$ & By Scalar Addition \\
\end{tabular}\\

$\therefore A(B+C) = AB + AC $
\begin{flushright}
$\blacksquare$
\end{flushright}

\end{document} 
