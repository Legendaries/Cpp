%----------------------------------------------------------------------------------------
%	SECTION A:  Document declaration and packages used for Latex Compile
%----------------------------------------------------------------------------------------
\documentclass{article}
%\documentclass[titlepage]{article}

\usepackage{graphicx} 						% Required for the inclusion of images
\usepackage{geometry}                		% See geometry.pdf to learn the layout options. There are lots.
\geometry{letterpaper}                   		% ... or a4paper or a5paper or ...
\usepackage{amssymb,amsmath}		% Math packages
\usepackage{epstopdf}							% encapsulated postscript to pdf
\usepackage{color}	
\usepackage{setspace}
\usepackage{etoolbox}
\usepackage{lipsum}
%----------------------------------------------------------------------------------------
%	SECTION B:  New commands defined for use in this document, examples given, you may change
%----------------------------------------------------------------------------------------
\newcommand{\blu}[1]{\textcolor{blue}{#1}}                     			% \blu{} will make your font blue
\newcommand{\red}[1]{\textcolor{red}{#1}}
\definecolor{light-gray}{gray}{0.8}
\definecolor{orange}{rgb}{1,0.5,0}
\newcommand{\gray}[1]{\textcolor{light-gray}{#1}}

%----------------------------------------------------------------------------------------
%	SECTION C:  Lab Title
%----------------------------------------------------------------------------------------
\title{FVCC Linear Algebra\\ Proof of The Distributive Property of Matrix Multiplication Over Addition \\ } 		% Title
\author{Daniel Church} 													% Author name
\date{\today} 	

\begin{document}
\maketitle
\noindent
% Lab Participants
\rule[4mm]{17cm}{2pt}
\begin{tabular}{l r}
Date Performed: & \today \\
Instructor: & M. Severino																% Instructor
\end{tabular}\\
\rule[4mm]{17cm}{2pt}
\AtBeginEnvironment{tabular}{\doublespacing}
%----------------------------------------------------------------------------------------
%	SECTION 1:  Description of what you intend to do in this lab
%----------------------------------------------------------------------------------------
\newpage
{\centering
\LARGE Proof of The Distributive Property of Matrix Multiplication Over Addition \par
}
\doublespacing
\begin{tabular}{@{}l l l @{\hskip 2.5in}l}
To Show: $A(B+C)$ & = & $AB + AC$ & 
\end{tabular}

Given $A, B$ and $C$ are Matrices of size $m$ x $n$, $n$ x $p$, and $n$ x $p$ respectively,

\begin{tabular}{@{}l l l @{\hskip 2in}l}
Let $D$ & = & $A+B$ & By Addition of Matrices\\
Then, $A(B+C)$ & = & $AD$ & By Substituion\\
$(AD)_{ij}$ & = & $\displaystyle\sum_{k=1}^{n} a_{ik}d_{kj}$ & By Def. of Matrix Multiplication\\%may remove the display style
$(AD)_{ij}$ & = & $\displaystyle\sum_{k=1}^{n} a_{ik}(b_{kj}+c_{kj})$ & By Substituion\\%may remove the display style
$(AD)_{ij}$ & = & $\displaystyle\sum_{k=1}^{n} a_{ik}b_{kj}+a_{ik}c_{kj}$ & By The Distrubutive Property\\%may remove the display style
& = & $\displaystyle\sum_{k=1}^{n} a_{ik}b_{kj} + \displaystyle\sum_{k=1}^{n} a_{ik}c_{kj}$ & \\
& = & $(AB)_{ij}+(AC)_{ij}$ & By Def. of Matrix Multiplication\\
& = & $(AB+AC)_{ij}$ & \\
\end{tabular}\\

$\therefore A(B+C) = AB + AC$
\end{document} 